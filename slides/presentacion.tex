\documentclass{beamer}
\usetheme[
          titleline=true,% Show a line below the frame title.
          titlepagelogo=ucmlogo,% Logo for the first page.
          logowidth=0.3,
          authorwidth=0.5,
          pageofpages = de
          ]{Zaragoza}

\author{Jorge Alda Gallo}
\title{Nuevas aplicaciones del modelo de Coleman-Weinberg}
\institute{Departamento de Física Teórica I, Universidad Complutense de Madrid}
\date{4 de Julio de 2016}
\logo{\includegraphics[height=30px]{ucmlogo}}
\usecolortheme{ucm}
\usepackage[utf8]{inputenc}

\setbeamertemplate{section in toc}[sections numbered]
\setbeamertemplate{subsection in toc}[sections numbered]
\usefonttheme[onlymath]{serif}

\begin{document}
\begin{frame}[t, plain]
\titlepage
\end{frame}

\begin{frame}[t]{Índice}
\tableofcontents
\end{frame}

\section{Introducción}
\begin{frame}[t]{Introducción}

\end{frame}
\subsection{El modelo electrodébil y el mecanismo de Higgs}
\begin{frame}[t]{El modelo electrodébil y el mecanismo de Higgs}

\end{frame}

\subsection{Problemas del mecanismo de Higgs}
\begin{frame}[t]{Problemas del mecanismo de Higgs}

\end{frame}

\subsection{El mecanismo de Coleman-Weinberg}
\begin{frame}[t]{El mecanismo de Coleman-Weinberg}

\end{frame}

\section{El modelo}
\begin{frame}[t]{El modelo}

\end{frame}

\section{Grupo de Renormalización}
\begin{frame}[t]{Grupo de Renormalización}

\end{frame}

\section{Mezcla de los escalares}
\begin{frame}[t]{Mezcla de los escalares}

\end{frame}

\section{Unificación de las constantes}
\begin{frame}[t]{Unificación de las constantes}

\end{frame}

\section{Fenomenología}
\begin{frame}[t]{Fenomenología}

\end{frame}

\subsection{Desintegraciones}
\begin{frame}[t]{Desintegraciones}

\end{frame}

\subsection{Producción en aceleradores}
\begin{frame}[t]{Producción en aceleradores}

\end{frame}

\section{Conclusiones}
\begin{frame}[t]{Conclusiones}

\end{frame}

\begin{frame}[t]{Índice}
\tableofcontents
\end{frame}

\end{document}

