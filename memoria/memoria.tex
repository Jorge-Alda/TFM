\documentclass[aps,prd,preprintnumbers,nofootinbibn,twocolumn]{revtex4}
%showpacs,superscriptaddress,groupedaddress
\usepackage{graphicx}  % needed for figures
\usepackage{dcolumn}   % needed for some tables
\usepackage{bm}        % for math
\usepackage{amssymb}   % for math
\usepackage[utf8]{inputenc}
%\usepackage[spanish]{babel}
\usepackage{hyperref}

\usepackage{amsmath}
\usepackage{array}
\usepackage{appendix}
\usepackage{epsfig}
\usepackage{siunitx}

\hyphenation{arXiv}

\addtolength{\textheight}{1.6cm}

\newcommand{\dif}{\mathrm{d}}
\makeatletter
\def\l@subsection#1#2{}
\def\l@subsubsection#1#2{}
\makeatother
\begin{document}


\title{NEW APPLICATIONS OF THE COLEMAN-WEINBERG MODEL\\[0.4cm]}
\thispagestyle{empty}


\date{de 2016}


\begin{abstract}
\begin{center}
\vspace{0.7cm}

{\bf Autor:}\\

\vspace{0.2cm}

Jorge Alda Gallo $^{(1)}$\\

\vspace{1cm}

{\bf Director:}\\

\vspace{0.2cm}

J. A. R. Cembranos $^{(2)}$\\

\vspace{0.6cm}


\vspace{5cm}
\begin{center}
\includegraphics[width=0.30\textwidth]{ucmlogo}
\end{center}
\vspace{5cm}
\end{center}

$^{(1)}$  E-mail: \href{mailto:jalda@ucm.es}{jalda@ucm.es}

$^{(2)}$ E-mail: \href{mailto:cembra@fis.ucm.es}{cembra@fis.ucm.es}
\vspace{8cm}

\end{abstract}




\clearpage


\maketitle

\begin{widetext}
\begin{quotation}

{\bf Resumen:}\\



\vspace{2cm}

{\bf Abstract:}\\



\vspace{2cm}
\end{quotation}
\end{widetext}
\tableofcontents


\vspace{5cm}




\newpage
\clearpage

\section{Introduction}
The Higgs boson has remained for a long time the elusive last piece of the standard model of particle physics (SM) to be discovered. The waiting came to an end on 2012, when it was announced the discovery at LHC \cite{Aad:2012tfa,Chatrchyan:2012xdj} of a new resonance at \SI{125}{\giga\electronvolt}.

But a \SI{125}{\giga\electronvolt} SM Higgs boson is not without problems. The coupling of the top quark with the Higgs might mean that the electroweak vacuum is metastable \cite{EliasMiro:2011aa}, and it would be able to decay to a lower energy vacuum. Naturalness indicates that the Higgs mass is far too low compared to the Planck scale, which leads to the so-called hierarchy problem \cite{Iso:2013aqa}. 


According to the SM, the Higgs field $H$ is characterized by the following potential
\begin{equation}
V(H) = m^2 H^\dagger H + \lambda_h (H^\dagger H)^2
\end{equation}
Here, $m^2 < 0$, which means that $H=0$ is unstable. Thus, the Higgs field has a vacuum expectation (vev) $v=\SI{246}{\giga\electronvolt}$ (known from masses of the $W$ and $Z$ bosons). This causes an spontaneous symmetry breaking that provides the masses for fermions and gauge bosons. The condition for the minimum of the potential is 
\begin{equation}
-m^2 = \lambda_h v^2
\end{equation}
The physical mass of the Higgs boson is determined as the value of the second derivative of the potential at the vacuum, that is, 
\begin{equation}
m_h^2 = 2 \lambda_h v^2
\end{equation}

An alternative proposal for the origin of the Higgs boson is due to Coleman and Weinberg \cite{Coleman:1973jx} (CW). The mass term is set to zero, and the symmetry breaking is caused by radiative corrections to the quartic term. In this way, the CW lagrangian is classically scale invariant and is free of the hierarchy problem. Unfortunately the CW mechanism is not without problems, for it predicts the opposite running of the Higgs quartic coupling.  

In the last couple of years, the CW mechanism has regained popularity, and some modifications have been proposed \cite{Dermisek:2013pta, Hill:2014mqa, Antipin:2015kgh}.  We will study a model where the CW is not realized by the Higgs boson itself, but by an additional scalar boson coupled to it. In section \ref{sect:model} we present this model.

\section{The model} \label{sect:model}

\bibliographystyle{apsrev4-1}
\bibliography{memoria} 

\newpage
\clearpage

\begin{widetext}
El abajo firmante, matriculado en el Máster de Física Teórica de la Facultad de Ciencias Físicas, autoriza a la Universidad Complutense de Madrid (UCM) a difundir y utilizar con fines académicos, no comerciales y mencionando expresamente a su autor el presente Trabajo de Fin de Máster: Nuevas aplicaciones del modelo de Coleman-Weinberg, realizado durante el curso académico 2015-2016 bajo la dirección de Jose A. Ruiz Cembranos en el Departamento de Física Teórica I, y a la Biblioteca de la UCM a depositarla en el Archivo institucional E-Prints Complutense u otra plataforma de la UCM que se cree para tal fin con el objeto de incrementar la difusión, uso e impacto del trabajo en Internet y garantizar su preservación y acceso a largo plazo.\\

La publicación en abierto tendrá un embargo de:
\\

\noindent$\,X\,$ Ninguno
\\


Fdo:\\


\vspace{3cm}

El abajo firmante, director del Máster de Física Teórica de la Facultad de Ciencias Físicas, autoriza a la Universidad Complutense de Madrid (UCM) a difundir y utilizar con fines académicos, no comerciales y mencionando expresamente a su autor el presente Trabajo de Fin de Máster: Nuevas aplicaciones del modelo de Coleman-Weinberg, realizado durante el curso académico 2015-2016 bajo mi dirección en el Departamento de Física Teórica I, y a la Biblioteca de la UCM a depositarla en el Archivo institucional E-Prints Complutense u otra plataforma de la UCM que se cree para tal fin con el objeto de incrementar la difusión, uso e impacto del trabajo en Internet y garantizar su preservación y acceso a largo plazo.\\

La publicación en abierto tendrá un embargo de:
\\

\noindent$\,X\,$ Ninguno
\\


Fdo:\\


\vspace{12cm}

\end{widetext}



\end{document} 
